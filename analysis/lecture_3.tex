\documentclass[a4paper, 11pt]{article}

\usepackage[margin=1in,letterpaper]{geometry}

\usepackage{amsmath}
\usepackage{amssymb}
\usepackage{amsthm}

\newtheorem{theorem}{Theorem}[section]
\newtheorem{corollary}{Corollary}[theorem]
\newtheorem{lemma}{Lemma}[theorem]
\newtheorem{proposition}{Proposition}[section]

\theoremstyle{definition}
\newtheorem{definition}{Definition}[section]
\newtheorem{example}{Example}[section]

\newtheoremstyle{break}
  {}%
  {}%
  {\itshape}
  {}%
  {\bfseries}
  {}%
  {\newline}%
  {}%

\theoremstyle{break}
\newtheorem{remark}{Remark}

\usepackage[parfill]{parskip}
\usepackage{bm}
\usepackage[flushleft]{threeparttable}

\usepackage{color}
\usepackage{comment}
\usepackage{cite}
\usepackage[final]{hyperref}
\hypersetup{
  colorlinks=true,
  linkcolor=blue,
  citecolor=blue,
  filecolor=magenta,
  urlcolor=blue
}

\DeclareMathOperator{\N}{\mathbb{N}}
\DeclareMathOperator{\R}{\mathbb{R}}
\DeclareMathOperator{\Q}{\mathbb{Q}}
\DeclareMathOperator*{\Z}{\mathbb{Z}}

\title{Existance of Nth Roots and Real Powers}

\begin{document}

\section{Difference of Powers formula}
Difference of Power is the generalization:
\begin{align}
  a^{n} - b^{n} &= (a - b)(a^{n - 1} + a^{n - 2}b + \cdots + ab^{n - 2} + b ^{n - 1}) \\
                &= (a - b) \sum_{k=0}^{n-1}a^{n- 1 - k }b ^{k}
\end{align}

where $\lbrace n : n \in \N n \text{ and } \geq 2 \rbrace$

In fact, we can use this result to show an upper bound and lower bound for $b^{n} - a^{n}$. If $0 < a < b$
then 
\begin{equation}
  (b - a)n a ^{n - 1} < b^{n} - a^{n} < (b - a)n b ^{n - 1}
\end{equation}

\begin{proof}
We assume that (2) is true and we try to obtain the left hand side of (1). then
\begin{align*}
  (a - b)\sum_{k=0}^{n-1}a^{n-1-k}b^{k} &= a \sum_{k=0}^{n -1}a^{n-1-k}b^{k} - b \sum_{k=0}^{n-1}a^{n-1-k}b^{k} \\
    &= \sum_{k=0}^{n -1}a^{n-k}b^{k} - \sum_{k=0}^{n-1}a^{n-1-k}b^{k+1} \\
    &= a^{n} + \left( \sum_{k=1}^{n -1}a^{n-k}b^{k} - \sum_{k=0}^{n-2}a^{n-1-k}b^{k+1}  \right) - b^{n} \\
    &= a^{n} - b^{n}
\end{align*}

So we are able to show the relation. Now we want to bound the for (3). We can do this by supposing 
$0 < a < b \in \R$ and that (2) holds.

To determine an upper bound, we replace a with b (since $b > a$). Multiply both sides of (2) by $-1$.
\begin{align*}
  (b - a)\sum_{k=0}^{n-1}b^{n-1-k}a^{k} &< (b - a) \sum_{k=0}^{n-1}b^{n-1-k}b^{k} \\
    &= (b - a) \sum_{k=0}^{n-1}b^{n-1} \\
    &= (b - a) n b^{n-1}
\end{align*}

So we see that $(b -a)nb ^{n-1}$ is an upper bound for $b^{n} - a^{n}$.

Without loss of generality we assume the reverse. For a lower bound we do the same thing but replace $b$ with $a$.
Then we see that $b^{n} - a^{n}$ can be bounded from below.

\end{proof}

\section{Bernoulli Inequality}
\section{Every Real number has $n^{th}$ root for all natural numbers}
\end{document}