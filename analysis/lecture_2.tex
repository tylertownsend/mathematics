\documentclass[a4paper, 11pt]{article}
\usepackage[margin=1in,letterpaper]{geometry}

\usepackage{amsmath}
\usepackage{amssymb}
\usepackage{amsthm}

\newtheorem{theorem}{Theorem}[section]
\newtheorem{corollary}{Corollary}[theorem]
\newtheorem{lemma}{Lemma}[theorem]
\newtheorem{proposition}{Proposition}[section]

\theoremstyle{definition}
\newtheorem{definition}{Definition}[section]
\newtheorem{example}{Example}[section]

\newtheoremstyle{break}
  {}%
  {}%
  {\itshape}
  {}%
  {\bfseries}
  {}%
  {\newline}%
  {}%

\theoremstyle{break}
\newtheorem{remark}{Remark}

\usepackage[parfill]{parskip}
\usepackage{bm}
\usepackage[flushleft]{threeparttable}

\usepackage{color}
\usepackage{comment}
\usepackage{cite}
\usepackage[final]{hyperref}
\hypersetup{
  colorlinks=true,
  linkcolor=blue,
  citecolor=blue,
  filecolor=magenta,
  urlcolor=blue
}


\DeclareMathOperator{\N}{\mathbb{N}}
\DeclareMathOperator{\R}{\mathbb{R}}
\DeclareMathOperator{\Q}{\mathbb{Q}}
\DeclareMathOperator*{\Z}{\mathbb{Z}}

\title{Completeness Axiom}

\begin{document}

\tableofcontents

\section{Power Inequality}
\begin{theorem}[Power Inequality]
For every $n \in \N$ and $x, y \in \R$ if $0 \leq x < y$, then $0 \leq x^{n} < y^{n}$
\end{theorem}

To prove this let us start with a lemma.
\begin{lemma}
Suppose $a, b, c \in \R$. If $a < b$ and $c > 0$, then $ac < bc$.
\end{lemma}
\begin{proof}
Suppose towards a contradiction that $ac = bc$. Then by the equality we require $bc \leq ac$. Since we have $c > 0$, then $c^{-1} > 0$. Then we have
\begin{equation*}
  b = bcc^{-1} \leq acc^{-1} = a
\end{equation*}
which contradicts the assumption that $a < b$.
\end{proof}

\begin{proof}
Let $P(n):$ for all $x, y \in \R$, if $0 \leq x < y$, then $x^{n} < y^{n}$.

\textbf{Base Case}\newline
Take $P(1): 0 \leq x < y$ which implies $0 \leq x^1 < y^1$. Which is true based on the given relationship of $x$ and $y$.

\textbf{Inductive Step}\newline
Suppose $P(n)$ holds, that is $\forall x, y, \in \R$, if $0\leq x < y$, then $0 \leq x^n < y^n$. Let $x, y \in R$, $0\leq x < y$.
Then 
\begin{align*}
  x^{n+1} &= x^n \cdot x \leq x^n \cdot y &\text{(using $x < y$, and $0 \leq x^n$)} \\
  &<y^n = y^{n+1} \cdot y &\text{(by the claim using $x^n < y^n$ and $0 < y$)}
\end{align*}
This then implies $x^{n+1} < y^{n+1}$ (1). Finally, since $0 \leq x^n$, and $0 \leq x$, then $0 \leq x^{n+1}$ (2). Taking these two (1 and 2)
we have the desired result.

\end{proof}

\section{The Real Numbers}
\subsection{Maxes and minimums}
Let $S \subseteq \R$, where $S \neq \emptyset$.

\begin{enumerate}
  \item The largest element of $S$ (if there is one), is called the \textit{maximum of S}, $\max S$
  \item The least element of $S$ (if there is one), is called the \textit{minimum of S}, $\min S$
\end{enumerate}

\begin{example}
Every finite non-empty $S \subseteq \R$ has a maximum and minimum.
\end{example}

\begin{example}
Let $a < b \in \R$. Then consider the closed interval
\begin{align*}
  [a, b] = \lbrace x \in \R: a \leq x \leq b \rbrace
\end{align*}
has minimum $\min [a, b] = a$ and maximum $\max [a, b] = b$.
\end{example}

\begin{example}
The open interval
\begin{align*}
  (a, b) = \lbrace x \in \R: a < x < b \rbrace
\end{align*}
has neither a max nor a min.
\end{example}

\begin{example}
$\Z, \Q, \subseteq \R$ do not have a min or max. Consider $a \in \Z$. Then there exists and $a + 1 \in \Z$ hence there is no max.
Conversely the same arg for min.
\end{example}

\subsection{Bounds on Sets}
\begin{definition}{Bounds}
Let $S \subseteq \R$ where $S \neq \emptyset$.
\begin{enumerate}
  \item We call $M \in \R$ \textbf{an upper bound for} $S$ if $M \geq S$ for all $s \in S$.
  \item We call $m \in \R$ \textbf{a lower bound for} $S$ if $m \leq S$ for all $s \in S$.
  \item We say that $S$ is bounded if $S$ is bounded from above and from below.
\end{enumerate}
\end{definition}

\begin{example}
The max of a set (if it exists) is an upper bound. The min of a set (if it exists) is a lower bound.
\end{example}

\begin{example}
Let $a < b \in \R$ then $a$ is a lower bound for $[a, b]$ and $(a, b)$. $b$ is an upper bound for
both sets as well.
\end{example}

\begin{example}
Neith of the set $\Z$ or $\Q$ are bounded from below or above.
\end{example}

\subsection{Supremum and Infinum}
\begin{definition}
Let $S \subseteq \R$ where $S \neq \emptyset$.
\begin{enumerate}
  \item If $S$ is bounded from above and has a least upper bound, $s_0$, then $s_0$ is the
  \textbf{supremum} of $S$, $s_0=\sup S$.
  \item If $S$ is bounded from below and has a greatest lower bound, $s_1$, then $s_1$ is the
  \textbf{infinum} of $S$, $s_1=\inf S$.
\end{enumerate}

\begin{remark}
  \begin{enumerate}
    \item Every $S \subseteq \R, S\neq \emptyset$, can have a most one supremum and one infinum.
    \item If $S \subseteq \R$ has a max, then $\max S = \sup S$.
    \item The following are equivalent:
    \begin{itemize}
      \item $s_0 = \sup S$
      \item $s_0 \geq s \forall s \in S$, and if $s_1 \geq s \forall s \in S,$ then $s_1 \geq s_0$
      \item $s_0 \geq s \forall s \in S$, and if $s_1 < s_0$, then $s_1 < s$ for some $s \in S$.
    \end{itemize}
  \end{enumerate}
\end{remark}
\end{definition}

\begin{example}
For $a < b \in \R$,
\begin{enumerate}
  \item $\sup[a, b] = \sup(a, b) = b$
  \item $\inf[a, b] = \inf(a, b) = a$
\end{enumerate}
\end{example}

\begin{example}
Let $A = \lbrace \frac{1}{n^{2}}: n \in \N, n\geq 3  \rbrace$ A is bounded from above and from below.
$\max A = 1 / 3$, however is there is no minimum.

\begin{enumerate}
  \item $\inf A = 0$. $A$ is the set of values where each value is equivalent to $1/n^2$ for values of 
  $n\geq 3$. So we can choose $n$ to be sufficiently large. Because $n \rightarrow \infty$,
  $\frac{1}{n^2} \rightarrow 0$. So we know that 0 is the greatest lowerbound.
  \item $\sup A = \max A = 1/3$
\end{enumerate}
\end{example}

\section{The completeness Axiom}
\begin{definition}{Completeness Axiom}
Every non-empty subset of $\R$ which is bounded from above has a least upper bound. This is equivalent
to: Given $S \subseteq \R$, $S \neq \emptyset$, if $S$ has at least one upper bound, then $\sup S$, exists.
\end{definition}

\begin{remark}
The Completeness Axiom failes for $\Q$.
\begin{align*}
  A &= \lbrace r \in \Q: 0 \leq r \text{ and } r^2 \leq 2 \rbrace \\
  &= \lbrace r \in \Q: 0 \leq r \leq \sqrt{2} \rbrace 
\end{align*}
\end{remark}
$A$ is bounded from above (e.g. $\frac{3}{2}$ is an upper bound), but the $\sup A = \sqrt{2}\not\in \Q$.

\begin{corollary}
Every $\emptyset \neq S \subseteq \R$ that is bounded from below has a greatest lower bound $\inf S$
\end{corollary}

\begin{proof}
Given $S \subseteq \R$, let $-S = \lbrace -s: s \in S \rbrace$. Given that S is bounded below, $\exists m\in \R$
such that $m \leq s$ for $s \in S$. This implies that $-m \geq -s$ for all $s \in S$.

So $-m \geq u, \, \forall u \in -S$. Thus $-S$ is bounded from above by $-m$. By the Completeness Axiom for
$-S$, the $\sup -S$ exists.

Let $s_0 = \sup -S$. What we need to show is that 
\begin{enumerate}
  \item $-s_0$ is a lower bound of $S$($-s_0 \leq s, \, \forall s \in S$)
  \item $-s_0$ is the greatest lower bound (if $t \leq s, \, \forall s \in S$ then $t \leq -s_0$)
\end{enumerate}.

We take $s_0 \geq -s$ for all $s \in S$. And this implies the condition (1) by multiply both sides
by -1.

For the second part assume $t \leq s, \, \forall s \in S$. This is equivalent to 
\begin{align*}
  &-t \geq -s, \, \forall s \in S\\
  \implies &-t \geq u, \forall u \in -S\\
  \implies &-t \geq s_0 \\
  \implies &t \leq s_0
\end{align*}
\end{proof}

\subsection{What is $\R$}
$\R$ is a number system containing $\Q$ and satisfying the Completeness Axiom.

\begin{definition}[Archimedian Property]
The following properties of $\R$ hold.
\begin{enumerate}
  \item For every positive real number $a > 0$, there is a natural number $n$ such that $n > a$.
  \item For every $a, b > 0$ in $\R$, there is an $n \in \N$ such that $n\cdot a > b$
  \item For every $\epsilon > 0 \in \R$ there is an $n \in \N$ such that $\frac{1}{n} < \epsilon$
\end{enumerate}
\end{definition}

\begin{proof}
\begin{enumerate}
  \item Assume towards contradiction that $\exists a > 0 \in \R$ such that $a \geq n$ for every $n \in \N$.
  Thus $a$ is an upper bound for $\N \subseteq \R$. By the completeness axiom, there exist some $b \in \R$
  such that $b = \sup \N$. As $b$ is the least upper bound of the set $\N$ the number $b - 1/2$ is
  not an upper bound for $\N$. In particular, $\exists n \in \N$ such that $n > b - \frac{1}{2}$.
  This implies that $n + 1 > b - \frac{1}{2} + 1 > b$. However, this says that $b \neq \sup N$,
  which is a contradiction.
  \item Suppose $a, b > 0$, in particular $\frac{b}{a} > 0$. By the first property, $\exists n\in \N$ such that $n > \frac{b}{a}$. This rearanging, $n\cdot a > b$.
  \item Suppose $\epsilon > 0, \, \epsilon \in \R$. Then $\frac{1}{\epsilon} > 0$. By (1) $\exists n \in \N$ such that math $n > \frac{1}{\epsilon}$. Rearanging, we can get $\epsilon > \frac{1}{n}$.
\end{enumerate}
\end{proof}

\begin{corollary}
  Suppose $a < b \in \R$ and $b - a > 1$. Then there is an integer $m$ such that $a < m < b$.
\end{corollary}

\begin{proof}
  By the Archimedian Property, $\exists k > \max (|a|, |b|)$. Then we know that $-k < a < b < k$. Let the sets $K = \lbrace j \in \Z: -k \leq j \leq k \rbrace$ and $K' = \lbrace j \in \Z: a \leq j \rbrace$ such that both $K, K'$ are finite and non empty as $k \in K'  \subseteq K$.
\end{proof}

\begin{theorem}[Density of $\Q$ in $\R$]
For every real numbers $a,b\in \R$ with $a<b$, $\exists r \in Q)$ such that $a < r < b$
\end{theorem}

\begin{proof}
  We need to find a quotient of integers, $m, n \in \Z$ such that $n > 0$ and
  \begin{align*}
    a < \frac{m}{n} < b
  \end{align*}
  We need to choose $n$ such that the demoniator is large enough so that consecutive increments of $1/n$ are too close to step over in the interval $(a, b)$.

  Using the Archimedean Property we may pick $\frac{1}{n} < \epsilon$ where $\epsilon = b - a$. 
\end{proof}

\end{document}