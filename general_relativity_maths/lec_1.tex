\documentclass[a4paper, 11pt]{article}
\usepackage[margin=1in,letterpaper]{geometry}

\usepackage{amsmath}
\usepackage{amssymb}
\usepackage{amsthm}

\newtheorem{theorem}{Theorem}[section]
\newtheorem{corollary}{Corollary}[theorem]
\newtheorem{lemma}{Lemma}[theorem]

\theoremstyle{definition}
\newtheorem{definition}{Definition}[section]
\newtheorem{example}{Example}[section]

\newtheoremstyle{break}
  {}%
  {}%
  {\itshape}
  {}%
  {\bfseries}
  {}%
  {\newline}%
  {}%

\theoremstyle{break}
\newtheorem{remark}{Remark}

\usepackage[parfill]{parskip}
\usepackage{bm}
\usepackage[flushleft]{threeparttable}

\usepackage{comment}
\usepackage{cite}
\usepackage[final]{hyperref}
\hypersetup{
  colorlinks=true,
  linkcolor=blue,
  citecolor=blue,
  filecolor=magenta,
  urlcolor=blue
}

\DeclareMathOperator{\N}{\mathbb{N}}
\DeclareMathOperator{\Otop}{\mathcal{O}}
\DeclareMathOperator{\Pset}{\mathcal{P}}
\DeclareMathOperator{\R}{\mathbb{R}}
\DeclareMathOperator{\Q}{\mathbb{Q}}
\DeclareMathOperator*{\Z}{\mathbb{Z}}

\title{Topology}

\begin{document}

\tableofcontents

\section{Introduction}
The Physical key definition underlying all modern physics.

\begin{center}
\fbox{%
  \parbox{0.9\textwidth}{%
  Spacetime is a four - \underline{dimensional} \underline{topological} \underline{manifold}
  with a \underline{smooth atlas} carrying a \underline{torsion-free}
  \underline{connection} compatible with a \underline{Lorentzian metric}
  and a \underline{time orientation} satisfying the \underline{Einstien Equations}
}}
\end{center}

\section{Topological Spaces}
Want to talk about space time and at the coarsest level, space time is a set. Each point is an element
of the set. However, this set is not enough to talk about continuity of maps.

Why as Physicist why do we want to talk about maps. In classical physics, curves do not jump. Particles
move along curves with no jumps. Structures on the set allows you to talk about the set.

Intrested in the Weakest structure we can establish on set which allows a good definition of continuitiy
of maps. As mathematicians we can establish the weakest structure on a set called a \textbf{topology}.

\begin{definition}{Topology}
  Let $M$ be a set. A topology $\Otop \subseteq \Pset(M)$ where $\Pset(M)$ is the power set of $M$. So you make a choice
  of sets to be be in $\Otop$. These should satisfy:
  \begin{enumerate}
    \item $\emptyset \in \Otop$ is required and $M \in \Otop$
    \item Let $U\in \Otop$ and $V \in \Otop$. Then the intersection of them is in $\Otop, \, U \bigcap V \in \Otop$.
    \item Assume $U_\alpha \in \Otop$ $\bigcup_{\alpha \in A} U_{\alpha} \in \Otop$
  \end{enumerate}
\end{definition}

\begin{example}
\begin{enumerate}
  \item $M = \{1, 2, 3\}$, $\Otop_1 = \{ \emptyset, \{1, 2, 3\} \}$ is a topology.
  \begin{itemize}
    \item The empty set is in $\Otop$ and $M \in \Otop$ so (1) qualifies.
    \item The intersection of both sets is the empty set so (2) checks out.
    \item The union of all sets is just (2) so (3) checks out.
  \end{itemize}
  \item $M = \{1, 2, 3\}$, $\Otop_2 = \{ \emptyset,\{1 \},\{ 2 \} \{1, 2, 3\} \}$ is not a topology.
\end{enumerate}
\end{example}

\begin{example}
Let $M$ be any set. Then
\begin{enumerate}
  \item The chaotic topology is defined as: \fbox{$\Otop_{chaotic} = \{\emptyset, M\}$}.
  \item The discrete topology is defined as: \fbox{$\Otop_{discrete} = \{\emptyset, \Pset(M)\}$}
\end{enumerate}
\end{example}

Why do we care about these? They are utterly useless :). They are extreme cases. The chaotic has the fewest possible 
and the discrete topology has the most elements. They serve as test cases for topologies.

There is a very important example which will help reconcile continuity, the standard topology.
\begin{definition}{The Standard Topology}
Let $M = \R^d = \R \times \R ... \times \R = \lbrace (p_1, ..., p_d): p_i \in \R \rbrace$. The standard topology is
$\Otop_{standard} \subseteq \Pset(\R^d)$.

The standard topology contains non-countable many elements. So we define
\begin{enumerate}
  \item The \textbf{softball}, $B_r(p)$, where $r \in \R^+$ and $p \in \R^d$ where 
  $B_r(p) = \lbrace (q_1, ..., q_d): \sum_{i = 1}^{d}(q_i - p_i)^2 < r^2 \rbrace$. You can think of $r$ as the radius of the
  ball and $p$ as the center of the ball. There is a softball for any $r$ and any $p$.
  \item $U \in \Otop_{standard}$ implies that $\forall p\in U \exists r \in \R^{+}$ where $B_r(p) \subseteq U$.
\end{enumerate}
\end{definition}

\begin{example}

\end{example}

A simple exercies why the standard topology satisfies these three axioms.

Eventually we are going to think of $M$ as points in spacetime. We defined topologies in order to define the continuitiy of maps.
Topologies are chosen to provide structure.

\section{Terminology}
\begin{table}[htpb]
  \centering
  \caption{caption}
  \begin{tabular}{ccc}
    \hline
    Symbol & Name & Definition \\
    \hline 
    $M$ & Set & (ZFC) \\
    \hline
    $\Otop$ & Topology & A set of open sets.\\
    \hline
    $(M, \Otop)$ & Topological Space & \\
    \hline
    $U\in \Otop, \, U \subseteq M$ & Open Set & \\
    \hline
    $M/A \in \Otop, \, A \subseteq M$ & Closed Set & \\
  \end{tabular}
\end{table}

\section{Continous Maps}
A map $f: M \rightarrow N$

\end{document}